\documentclass[
11pt, 
DIV10,
a4paper, 
oneside, 
headings=normal, 
captions=tableheading,
final, 
numbers=noenddot
]{scrartcl}

\usepackage{lipsum}
\usepackage{graphicx}
\usepackage{amsmath}
\usepackage{mathtools}

\newcommand*\vechs[3]{
	\begin{pmatr3ix}
		#1\\
		#2\\	
		#3\\
	\end{pmatrix}
}
\newcommand*\vechb[4]{
	\begin{pmatrix}
		#1\\
		#2\\	
		#3\\
		#4\\
	\end{pmatrix}
}

%\newcommand*{\bm}{\boldsymbol}
\newcommand*\ve[1]{\vec{v}_{#1}}



\title{Thema}
\subtitle{\vspace{0.5cm}Seminar: Current Topics in Physically-Based Animation}
\author{Autor}


\begin{document}
\tableofcontents
\section{General Idea/Introduction}
\section{Related Work}
\section{Method}
	\subsection{Rigid Body Simulation}
	\subsection{Skinning}
	\subsection{Interpolation between Examples}
		\subsubsection{Quaterions}
		\subsubsection{Quaternionrotation}
		\subsubsection{QLERP}
	\subsection{Impulse Projection on Example Poses}
	\subsection{Propagation of the Deformations}
	\subsection{Application over time}
	\subsection{Restitution Modification}
\section{Results and Discussion}
	\subsection{Performance}
	\subsection{Limitations and  future Work}
	
\newpage
\section*{Fragen}
\subsection*{Methodisch}
Was genau ist Ziel der Seminarsarbeit?\\
Soll ich das ganze Paper wiedergeben, oder nur die Methode erklären?\\
Gibt es inhaltliche Unterschiede zwischen Paper und Vortrag? Welche inhaltlichen gibt es?\\
Soll ich Rigid Body Simulation und Collision Detection genauer eklaeren?\\
Wie soll ich alles in 10 Seiten packen? Ist ein Layout, wie bei Papers besser?
\subsection*{Inhaltlich}
Auf Seite 4: Warum wird der Inhalt des Minimums quadriert?\\
Warum ist (6) die gradient direction fuer das Minimierungsproblem?
Was genau ist mit a few dijkstra passes gemeint?

\end{document}          
